\documentclass[aspectratio=169]{beamer}
\usetheme{Boadilla} % plainest one with slide number footer

% generic packages

\usepackage[utf8]{inputenc}
\usepackage[english]{babel}
\usepackage{amsmath}
\usepackage{multicol}

% about the presentation
\title{CoAP Protocol Indication}
\hypersetup{pdftitle={CoAP Protocol Indication}}
\subtitle{\texttt{draft-amsuess-core-transport-indication-01}\\\footnotesize which you may know from having piggybacked on resource-directory-extensions on IETF110}
\author{Christian~Amsüss}
\date{2021-07-28, IETF111}

% used commands

\usepackage{verbatim}

\definecolor{darkgreen}{rgb}{0, 0.56, 0}

% attach self

\usepackage{embedfile}
\embedfile{\jobname.tex}

\begin{document}

\frame{\titlepage}

\begin{frame}{A brief history of CoAP schemes}\large
	\begin{tabular}{r@{\hspace{1cm}}ll}
		2014 & RFC7252 & \texttt{coap} for UDP \\
		2015--2017 & \texttt{coap-tcp-tls$\mathtt{\le}$08} & \texttt{coap+tcp} for TCP \\
		2017 & \texttt{coap-tcp-tls-09} & \texttt{coap} for TCP or UDP \\
		     & \texttt{coap-tcp-tls$\mathtt{\ge}$10} & \texttt{coap+tcp} for TCP
	\end{tabular}

	\pause
	\bigskip

	\begin{block}{From No-Objection ballots}
		\ldots these scheme registrations [\ldots] present an ``antipattern'' \ldots

		\smallskip

		This runs counter to the principle that a URI identifies a resource \ldots

		\smallskip

		I am perplexed that no concrete mechanism for UDP/TCP failover is provided \ldots
	\end{block}
\end{frame}

\setbeamertemplate{background}{\includegraphics[page=42,width=\textwidth]{89.pdf}}
\begin{frame}{It was known not to be easy}
	\framesubtitle{From Bill's 2014 (IETF 89) \href{https://www.ietf.org/proceedings/89/slides/slides-89-core-0.pdf}{presentation}}
\end{frame}
\setbeamertemplate{background}{}

\begin{frame}{URI aliasing pain points}\Large

	\texttt{coaps://[2001:db8::1]/cfe} $\stackrel{?}{=}$ \texttt{coaps+tcp://[2001:db8::1]/cfe}

	\center Pick one side:
	\begin{multicols}{2}
	\begin{itemize}
		\item Multiple entries in discovery
		\item Multiple entries in caches
	\end{itemize}

	\begin{itemize}
		\item Transports stay unused
		\item Devices can not connect\footnote{CoAP-over-UDP was never described as mandatory-to-implement}
	\end{itemize}
	\end{multicols}

	\raggedright

	\bigskip

	Proxies can't pick transports according to their abilities.

	\bigskip

	No established terminology to describe URI aliasing.
\end{frame}

\begin{frame}[fragile]{Tools we have}\Large
	\begin{itemize}
		\item Easy resource metadata
			\begin{verbatim}
			 <coap+tcp://[2001:db8::1]>;rel=...
			\end{verbatim}%
			{\footnotesize (with some indirection to make site-wide statements)}
			\bigskip
		\item<2-> Cheap proxying

			\texttt{
			 CON GET\\
			~Observe: 0\\
			~Uri-Path: "cfe"\\
			}\color{darkgreen}\texttt{%
			+Proxy-Scheme: "coap"
			}
			\color{black}

			{\footnotesize (with triviality bonus points for implementations ignoring the 'critical' flag)}
	\end{itemize}
\end{frame}

\begin{frame}{Putting it together}
	\mbox{}\\
	\texttt{</cfe>;rt="tag:...:coffemachine"}{\color{gray}\texttt{;rel=hosts;anchor="/"}}\texttt{,}\\
	\texttt{<coap+tcp://[2001:db8::1]>;rel=}{\color{blue}\texttt{has-proxy}}{\color{gray}\texttt{;anchor="/"}}
	\begin{block}{Goals (1-2/5)}
		\textbf{Enablement} Inform clients of the availability of other transports of servers.

		\textbf{No Aliasing} Any URI aliasing must be opt-in by the server. Any defined mechanisms must allow applications to keep working on the canonical URIs given by the server.
	\end{block}

	\bigskip

	Server implementation: Just accept provided Proxy-Scheme options.

	\bigskip

	Client implementation: Ignore, or use indicated protocol and add Proxy-Scheme (and, if needed, Uri-Host) option.
\end{frame}

\begin{frame}{Message overhead kills}\large
	{\texttt{
	 CON GET\\
	~Observe: 0\\
	~Uri-Path: "cfe"\\
	}\color{darkgreen}\texttt{%
	+Proxy-Scheme: "coap"
	}}

	\bigskip

	$\sim 5$ bytes per request. More if host names are involved.

	\begin{block}{Goals (3/5)}
		\textbf{Optimization} Do not incur per-request overhead from switching protocls. This may depend on the server's willingness to create aliased URIs.
	\end{block}

	\texttt{rel=}{\color{blue}\texttt{has-unique-proxy}} additionally means you can skip Proxy-Scheme and Uri-Host
\end{frame}

\begin{frame}{Proxy interaction}\large
	\begin{block}{Goals (4-5/5)}
		\textbf{Proxy usability} All information provided must be usable by aware proxies to reduce the need for duplicate cache entries.

		\textbf{Proxy announcement} Allow third parties to announce that they provide alternative transports to a host.
	\end{block}

	\bigskip

	\ldots which I'll be happy to elaborate on in hallway discussions.
\end{frame}

\begin{frame}{Security Considerations}
	\Huge Just As With Any Proxy.

	\bigskip

	\footnotesize OK, there's more in the text, but that's the gist.

	\vspace{3cm}

	\large Problematic with third-party protocol translation services:

	What's done by (D)TLS users here? Do they use proxies at all?

	Are all-valid certificates common there? Do we want to endorse them?
\end{frame}

\begin{frame}{Take-home message}\Large 
	\begin{itemize}
		\item It can probably be just this simple.
		\item No URI aliasing introduced in applications.
	\end{itemize}

	\vspace{2cm}

	Questions? Comments? Interest?
\end{frame}

\begin{frame}{Backup slide / FAQ}
	\setlength{\parskip}{0.5em}
	\textit{Didn't we want to do this with DNS?}

	We\footnote{Whoever wants to use it will need to volunteer as coauthor.} still can, just need to phrase the equivalent statements in DNS.

	Straw man for ``\texttt{coap://device.example.com} has CoAP-over-TCP running on port 1234'':

	\texttt{\_has-coap-proxy.\_tcp.device.example.com SRV 0 0 device.example.com 1234}
	\texttt{device.example.com AAAA 2001:db8::1}

	\bigskip

	\textit{How does this relate to HTTP's \texttt{Alt-Svc}?}

	Generally similar; links instead of headers (as common in CoAP), and no need for protocol-id because we have schemes already.
\end{frame}

\end{document}
